\documentclass[12pt]{article}
\usepackage[left=2cm, right=2cm, top=3cm, bottom=3cm]{geometry}
\usepackage{fontspec}
\usepackage{url}
\usepackage{parskip} 

\setmainfont{Barlow} % Cambiar la fuente principal a Arial

\usepackage{listings}
\usepackage{color}
\definecolor{dkgreen}{rgb}{0,0.6,0}
\definecolor{gray}{rgb}{0.5,0.5,0.5}
\definecolor{mauve}{rgb}{0.58,0,0.82}
\lstset{frame=tb,
  language=Java,
  aboveskip=3mm,
  belowskip=3mm,
  showstringspaces=false,
  columns=flexible,
  basicstyle={\small\ttfamily},
  numbers=none,
  numberstyle=\tiny\color{gray},
  keywordstyle=\color{blue},
  commentstyle=\color{dkgreen},
  stringstyle=\color{mauve},
  breaklines=true,
  breakatwhitespace=true,
  tabsize=3
}

\title{Revisión de una Metodología para la Extracción, Limpieza y Organización de Datos SQL con Programación Declarativa en Python para el Desarrollo de Análisis de Datos Descriptivos.}

\begin{document}
\maketitle

\section{Procedimientos ETL}

Recordemos que los procedimientos ETL constan de Extract, Transform y Load, los cuales nos dan la pauta a seguir para seguir el ciclo de la Ingeniería de Datos. De esta forma seguimos una metodología establecida y evitamos estar trabajando sin una dirección específica.

\vspace{12pt}

En este ejemplo, se utilizará la base de datos de Sakila, la cual es una base de datos ejemplo proporcionada por MySQL que es utilizada comúnmente para propósitos educativos, demostraciones y pruebas. Sakila simula una base de datos de una tienda de alquiler de películas, similar a la cadena de alquiler de vídeos Blockbuster que solía existir.

\vspace{12pt}

Sakila contiene tablas que representan películas, actores, clientes, tiendas, alquileres, etc.

\vspace{12pt}

Para hacer uso de ella, hemos descargado el archivo script SQL a través de la página oficial de MySQL: \url{https://dev.mysql.com/doc/index-other.html}.

\vspace{12pt}

Ahora para instalar la base de datos podremos utilizar los siguientes comandos de MySQL.

\vspace{12pt}
\begin{lstlisting}[language=Bash]
    
    $ mysql -u root -p sakila < sakila-schema.sql
    $ mysql -u root -p sakila < sakila-data.sql
    
\end{lstlisting}
\vspace{12pt}

Después de haber ejecutado estos comandos, deberíamos ser capaces de ver la base de datos instalada en nuestro sistema, para comprobarlo podemos ejecutar:

\vspace{12pt}
\begin{lstlisting}[language=Bash]
    
  mysql> USE sakila;
  Database changed
  mysql> SHOW TABLES;
  +----------------------------+
  | Tables_in_sakila                |
  +----------------------------+
  | actor                           |
  | actor_info                      |
  | address                         |
  | category                        |
  | city                            |
  | country                         |
  | customer                        |
  | customer_list                   |
  | film                            |
  | film_actor                      |
  | film_category                   |
  | film_list                       |
  | film_text                       |
  | inventory                       |
  | language                        |
  | nicer_but_slower_film_list      |
  | payment                         |
  | rental                          |
  | sales_by_film_category          |
  | sales_by_store                  |
  | staff                           |
  | staff_list                      |
  | store                           |
  +----------------------------+

    
\end{lstlisting}
\vspace{12pt}

Ahora, con ello podremos comenzar a trabajar nuestro procedimiento ETL.

Primeramente, es necesario determinar que incógnita queremos responder sobre los datos disponibles. Esto variará dependiendo de la base de datos y los tipos de datos que ésta almacene. Para este caso, intentaremos responder las preguntas:

\begin{itemize}
  \item \textbf{Patrones de alquier:} ¿Cuáles son las películas más alquiladas?
  \item \textbf{Clientes frecuentes:} ¿Quiénes son los clientes más frecuentes?, ¿Existe alguna relación entre la cantidad de películas alquiladas y la ubicación de las tiendas?
  \item \textbf{Popularidad de las categorías de películas:} ¿Cuáles son las categorías de películas más populares entre los clientes?
  \item \textbf{Ingresos por película:} ¿Cuáles son las películas que generan más ingresos en términos de alquiler?
\end{itemize}

\subsection{Extract.}

Para nuestro ejemplo, utilizaremos Python con SQL Alchemy.

\vspace{12pt}
\begin{lstlisting}[language=Python]
    
    # utilizar sqlalchemy
    
\end{lstlisting}
\vspace{12pt}

\subsection{Transform.}

Ahora bien, después de haber extraído los datos, es necesario organizarlos de manera que sea más sencillo utilizarlos para el análisis de datos.

\vspace{12pt}
\begin{lstlisting}[language=Python]
    
    # utilizar pandas
    
\end{lstlisting}
\vspace{12pt}

\subsection{Load.}

Una vez que se haya realizado la Organización de los datos, es una buena idea es crear un acceso a esta información de manera que no sea necesario acceder a la información y organizarla nuevamente, para esto son muy útiles las vistas (views) de SQL.

\vspace{12pt}
\begin{lstlisting}[language=Python]
    
    # utilizar sqlalchemy
    
\end{lstlisting}
\vspace{12pt}



\end{document}
